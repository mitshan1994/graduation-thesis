% -*- coding: utf-8 -*-
\chapter{$T_3$仿紧空间的一些刻画}
\label{chap03}

接下来, 我们进入本文的正题, 就是对仿紧空间(附加上一些分离公理)的刻画,
使仿紧性质更加清晰一些, 而不仅仅只是个定义.

\begin{thm} \label{thm:paracompact sigma finite}
  设X是$T_3$空间,则下面各条件等价:
  \begin{enumerate}
  \item X是仿紧的;
  \item X的任意开覆盖有$\sigma$-局部有限开加细;
  \item X的任意开覆盖有局部有限加细;
  \item X的任意开覆盖有局部有限闭加细.
  \end{enumerate}
\end{thm}
\begin{proof}
  %% PROOF HERE
  (1) $\Rightarrow$ (2) :
  由仿紧的定义, 是显然的.
  
  (2) $\Rightarrow$ (3) :
  设$\mathscr{U} = \bigcup_{i = 0}^{\infty} \mathscr{U}_i$,
  其中$\mathscr{U}_i$是局部有限(非空)开集族.
  对$\forall i \in \mathbb{N}$, $\forall U \in \mathscr{U}_i$, 记
  \[
    C(i, U) = U \setminus \bigcup_{k = 0}^{i - 1} \mathscr{U}_k,
  \]
  \[
    \mathscr{V}_i = \{ C(i, U) \mid U \in \mathscr{U}_i \}.
  \]
  其中$\mathscr{V}_0 = \mathscr{U}_0$.
  下面证$\mathscr{V} = \bigcup_{i = 0}^{\infty} \mathscr{V}_i$是$\mathscr{U}$的局部有限加细.
  对$\forall x \in X$,
  记$n(x) = \text{min}\{ i \in \mathbb{N} \mid x \in \bigcup \mathscr{U}_i \}$.
  即存在$U_x \in \mathscr{U}_{n(x)}$, 使得$x \in U_x$.
  所以有$x \in C(n(x), U_x)$, $\bigcup \mathscr{V} = X$, $\mathscr{V}$是$\mathscr{U}$的加细.
  接着, 对$\forall i > n(x)$, 有$U_x \cap \bigcup V_i = \varnothing$.
  对$\forall i \leq n(x), \exists x$的领域$A_i$,
  最多只与$\mathscr{U}_i$中的有限个元素相交,
  因此也最多只与$\mathscr{V}_i$中的有限个元素相交.
  所以$x$存在邻域$U_x \cap \bigcap_{i = 0}^{n(x)} A_i$最多和$\mathscr{V}$中的有限个元素相交.

  (3) $\Rightarrow$ (4) :
  设$\mathscr{U}$是X的一个开覆盖.
  对$\forall x \in X, \exists U_x \in \mathscr{U},
  $使得$x \in U_x, $则$x \notin U_x^c, U_x^c$为闭集.
  再由$X$满足$T_3$公理, $\exists $开集$V_x$, 使得$\overline{V_x} \subseteq U_x$.
  $\mathscr{V} = \{ V_x \mid x \in X \}$是$X$的一个开覆盖,
  而$\mathscr{V}^{'} = \{ \overline{V} \mid V \in \mathscr{V} \}$是$\mathscr{U}$的加细.
  由条件, $\mathscr{V}$有局部有限加细$\mathscr{W} = \{ W_j \mid j \in J \}$.
  因为$\mathscr{V}^{'}$是闭覆盖,
  所以$\mathscr{W}^{'} = \{ \overline{W} \mid W \in \mathscr{W} \}$
  是$\mathscr{V}^{'}$的局部有限闭加细, 从而也是$\mathscr{U}$的局部有限闭加细.
  
  (4) $\Rightarrow$ (1) :
  设$\mathscr{A}$为$X$的一个开覆盖, 由条件,
  $\mathscr{A}$存在局部有限闭加细$\mathscr{B}$.
  对$\forall x \in X, \exists x$的邻域$C_x$, 只与$\mathscr{B}$中有限个元素相交.
  记$\mathscr{C} = \{ C_x \mid x \in X \}$, 则$\mathscr{C}$为$X$的一个开覆盖.
  再由条件, $\mathscr{C}$存在局部有限闭加细$\mathscr{D}$.
  对$\forall B \in \mathscr{B},$ 
  记$B^{'} = X \setminus \bigcup \{ D \in \mathscr{D} \mid D \cap B = \varnothing \}$.
  能看出, $B^{'}$为开集且$B \subseteq B^{'}$.
  对$\forall B \in \mathscr{B}, D \in \mathscr{D}$,
  我们有$B^{'} \cap D = \varnothing$当且仅当$B \cap D = \varnothing$.
  对$\forall B \in \mathscr{B}, \exists A_B \in \mathscr{A}$,
  使得$B \subseteq A_B$. 记$\mathscr{U} = \{ B^{'} \cap A_B \mid B \in \mathscr{B} \}$.
  因为对$\forall B \in \mathscr{B}, \exists U \in \mathscr{U},$ 
  使得$B \subseteq U$, 所以$\mathscr{U}$为$X$的一个开覆盖, 
  从而$\mathscr{U}$是$\mathscr{A}$的一个开加细.
  而$\forall D \in \mathscr{D}, D$只与$\mathscr{U}$中有限个元素相交.
  加上$\mathscr{D}$是$X$的覆盖, 得出$\mathscr{U}$是局部有限的.
  所以$\mathscr{U}$是$\mathscr{A}$的局部有限开加细, 所以$X$是仿紧的.
\end{proof}

这个定理说明了, 局部有限开加细其实是一个比较强的条件, 是可以减弱的,
比如是$\sigma$-局部有限的开加细就可以了. 对于一些特定的情况下,
这个定理对于判断某些空间是不是仿紧的, 就比较方便了.

上面的条件还可以进一步减弱一些. 因为局部有限或者$\sigma$-局部有限,
同闭包保持比起来, 可能看上去还是要强一些的, 因为前者涉及到局部的性质,
而后者并没有在明处涉及到.

下面我们来看一看闭包保持(开/闭)加细和仿紧之间的联系.

\begin{thm} \label{thm:closure-preserving}
  设X是正则的Hausdorff空间,则下面各条件等价:
  \begin{enumerate}
  \item X是仿紧的;
  \item X的任意开覆盖有闭包保持开加细;
  \item X的任意开覆盖有闭包保持加细;
  \item X的任意开覆盖有闭包保持闭加细.
  \end{enumerate}
\end{thm}

该定理证明的主要部分是(4) $\Rightarrow$ (1), 而(1) $\Rightarrow$ (2),
(2) $\Rightarrow$ (3), (3) $\Rightarrow$ (4)相对来说容易一些.
为了证明(4) $\Rightarrow$ (1), 我们需要先证明几个引理.
为了方便, 我们把频繁出现的条件列出来, 避免重复说明.

\begin{condition} \label{condition:closure-preserving}
  空间$X$是正则Hausdorff空间, 且$X$的任意开覆盖都有闭包保持闭加细.
\end{condition}

\begin{lemma} \label{lemma:indexed closure-preserving}
  在条件\ref{condition:closure-preserving} 下,
  如果$\{ U_{\alpha} \}_{\alpha \in A}$是X的一个开覆盖,
  则X存在一个闭包保持闭覆盖$\{ C_{\alpha} \}_{\alpha \in A}$,
  使得$\forall \alpha \in A, C_{\alpha} \subseteq U_{\alpha}$.
\end{lemma}
\begin{proof}
  %% PROOF HERE
  由条件, $\{ U_{\alpha} \}_{\alpha \in A}$存在一个闭包保持闭加细$\mathscr{V}$.
  对$\forall V \in \mathscr{V},$ 选取$\alpha(V) \in A$, 使得$V \subseteq U_{\alpha(V)}$.
  对$\forall \alpha \in A$, 定义
  \[
    C_{\alpha} = \bigcup \{ V \in \mathscr{V} \mid \alpha(V) = \alpha \}.
  \]
  则$C_{\alpha} \subseteq U_{\alpha}$. 由$\mathscr{V}$是闭包保持闭加细,
  $C_{\alpha}$是闭集, 且$\{ C_{\alpha} \}_{\alpha \in A}$是闭包保持的.
\end{proof}

\begin{lemma} \label{lemma:regular}
  在条件\ref{condition:closure-preserving} 下, X是正规的.
\end{lemma}
\begin{proof}
  %% PROOF HERE
  设$E_1, E_2$是$X$中的两个互不相交的闭集,
  则$\{ X \setminus E_1, X \setminus E_2 \}$是X的一个开覆盖.
  由引理\ref{lemma:indexed closure-preserving},
  $X$存在闭覆盖$\{ C_1, C_2 \}$,
  使得$C_1 \subseteq X \setminus E_1, C_2 \subseteq X \setminus E_2$.
  则$E_1 \subseteq X \setminus C_1, E_2 \subseteq X \setminus C_2$,
  而$(X \setminus C_1) \cap (X \setminus C_2) = \varnothing$,
  且都是开集. 所以X是正规的.
\end{proof}

\begin{lemma} \label{lemma:Dowker}
  \text{(Dowker).} 设X是正规的, $\{ V_\gamma \}_{\gamma \in \Gamma}$是X的离散的开子集族.
  如果对$\forall \gamma \in \Gamma, \text{有}D_{\gamma} \subseteq V_{\gamma}$,
  且$\bigcup_{\gamma \in \Gamma} D_{\gamma}$是闭集,
  则X存在一个离散的开子集族$\{ G_{\gamma} \}_{\gamma \in \Gamma}$, 使得
  \[
    D_{\gamma} \subseteq G_{\gamma} \subseteq V_{\gamma}, \,\,\, \forall \gamma \in \Gamma.
  \]
\end{lemma}
\begin{proof}
  %% PROOF HERE
  记
  \[
    A = \{ x \in X \mid x \text{存在邻域至多与一个} V_{\gamma} \text{相交非空} \}.
  \]
  容易得到$A$为开集, 且
  $\bigcup_{\gamma \in \Gamma} D_\gamma \subseteq A$.
  因为$\bigcup_{\gamma \in \Gamma} D_\gamma$是闭集, $X$是正规的,
  所以存在$X$中开集$B$, 使得
  \[
    \bigcup_{\gamma \in \Gamma} D_\gamma \subseteq B \subseteq \overline{B} \subseteq A.
  \]
  令
  \[
    G_\gamma = V_\gamma \cap B.
  \]
  则$G_\gamma$是开集, 且对$\forall \gamma \in \Gamma,
  D_\gamma \subseteq G_\gamma \subseteq V_\gamma$.
  由$\{ V_\gamma \}_{\gamma \in \Gamma}$是离散的, $\{ G_\gamma \}_{\gamma \in \Gamma}$也是离散的.
\end{proof}

有了上面的准备, 我们终于可以开始这个定理的证明了.

\begin{proof}[证明(定理\ref{thm:closure-preserving})]
  %% PROOF HERE
  (4) $\Rightarrow$ (1): 设$\{ U_{\alpha} \}_{\alpha \in A}$是$X$的一个开覆盖, 且$A$是一个良序指标集.
  我们需要证明的是$\{ U_{\alpha} \}_{\alpha}$存在$\sigma$-局部有限开加细.
  
  \underline{Step 1}.
  我们先构造一列子集族, 对$\forall i \in \mathbb{N}$,
  $\{ C_{\alpha, i} \}$满足下面的条件: \newline
  \textcircled{a}
    $\{ C_{\alpha, i} \}_{\alpha \in A}$是$X$的闭包保持闭覆盖,
    且对$\forall \alpha \in A, C_{\alpha, i} \subseteq U_{\alpha}$.\newline
  \textcircled{b}
    $C_{\alpha, i+1} \cap C_{\beta, i} = \varnothing,$ 对$\forall \alpha > \beta$.\newline
  构造方法如下. 由引理\ref{lemma:indexed closure-preserving},
  存在$\{ C_{\alpha, 0} \}_{\alpha \in A}$满足\textcircled{a}(自然地满足\textcircled{b}),
  此时$i = 0$. 假设对$i \leq n, \{ C_{\alpha, i} \}_{\alpha \in A}$均满足
  \textcircled{a}和\textcircled{b}. 下面我们构造$\{ C_{\alpha, n+1} \}_{\alpha \in A}$.
  我们定义
  \[
    U_{\alpha, n+1} = U_\alpha - ( \bigcup_{\beta < \alpha} C_{\beta, n} ),\,\,\,\forall \alpha \in A.
  \]
  因为$\{ C_{\alpha, n} \}_{\alpha \in A}$是闭包保持的, 所以$U_{\alpha, n+1}$为开集.
  对$\forall x \in X$, 存在最小的$\alpha_0$, 使得$x \in U_{\alpha_0}$,
  同时, $x \in U_{\alpha_0, n+1}$, 所以$\{ U_{\alpha, n+1} \}_{\alpha \in A}$是X的一个开覆盖.
  再由引理\ref{lemma:indexed closure-preserving},
  存在一个闭包保持闭覆盖$\{ C_{\alpha, n+1} \}_{\alpha \in A}$,
  使得对$\forall \alpha \in A, C_{\alpha, n+1} \subseteq U_{\alpha, n+1}$,
  从而$C_{\alpha, n+1} \subseteq U_{\alpha}$, 因此\textcircled{a}满足.
  由$U_{\alpha, n+1}$的定义,
  有$U_{\alpha, n+1} \cap C_{\beta, n} = \varnothing, \forall \beta < \alpha$.
  因为$C_{\alpha, n+1} \subseteq U_{\alpha, n+1}$, 
  所以$C_{\alpha, n+1} \cap C_{\beta, n} = \varnothing, \forall \beta < \alpha$.
  \textcircled{b}也满足了.构造成功.

  \underline{Step 2}.
  我们继续构造一列子集族. 对$\forall \alpha \in A, i \in \mathbb{N}$, 定义
  \[
    V_{\alpha, i} = X \setminus (\bigcup_{\beta \neq \alpha} C_{\beta, i}).
  \]
  我们有\newline
  \textcircled{c}
    $\{ V_{\alpha, i} \mid \alpha \in A, i \in \mathbb{N} \}$是$X$的一个开覆盖,
    且对$\forall \alpha \in A, i \in \mathbb{N}, V_{\alpha, i} \subseteq U_\alpha$. \newline
  \textcircled{d}
    $V_{\alpha, i} \cap V_{\beta, i} = \varnothing$, 当$\alpha \neq \beta$时. \newline
  因为$\{ C_{\alpha, i} \}_{\alpha \in A}$是闭包保持的闭覆盖, 所以$V_{\alpha, i}$是开集,
  且我们有$V_{\alpha, i} \subseteq C_{\alpha, i} \subseteq U_{\alpha, i}, \forall \alpha, i$.
  再由$V_{\alpha, i}$的定义, 可知\textcircled{d}满足.
  对于\textcircled{c}, 我们还需要证明
  $\{ V_{\alpha, i} \mid \alpha \in A, i \in \mathbb{N} \}$是$X$的一个覆盖.
  任取$x \in X$, 由集合$A$是良序的, 令
  \[
    \alpha_i = \min \{ \alpha \in A \mid x \in C_{\alpha, i} \}, \,\,\, i \in \mathbb{N}.
  \]
  再选取一个自然数$k$, 使得
  \[
    \alpha_k = \min \{ \alpha_i \mid i \in \mathbb{N} \}.
  \]
  由定义, 对$\forall \alpha < \alpha_k, x \notin C_{\alpha, i}$.
  对$\forall \alpha > \alpha_k$, 由\textcircled{b},
  $C_{\alpha, k+1} \cap C_{\alpha_k, k} = \varnothing$,
  所以$x \notin C_{\alpha, k+1}$. 但由于$\{ C_{\alpha, k+1} \}_{\alpha \in A}$是$X$的一个覆盖,
  所以$x$属于$C_{\alpha_k, k+1}$. 再由$V_{\alpha_k, k+1}$的定义,
  有$x \in V_{\alpha_k, k+1}$.
  所以$\{ V_{\alpha, i} \mid \alpha \in A, i \in \mathbb{N} \}$是$X$的覆盖.

  \underline{Step 3}.
  由引理\ref{lemma:indexed closure-preserving},
  存在$X$的闭包保持闭覆盖$\{ D_{\alpha, i} \mid \alpha \in A, i \in \mathbb{N}$,
  使得对$\forall \alpha, i, D_{\alpha, i} \subseteq V_{\alpha, i}$.
  由引理\ref{lemma:regular}, X是正规的, 对$\forall i$, 由引理\ref{lemma:Dowker},
  存在离散开子集族$\{ G_{\alpha, i} \}_{\alpha \in A}$, 使得
  \[
    D_{\alpha, i} \subseteq G_{\alpha, i} \subseteq V_{\alpha, i}, \,\,\, \forall \alpha.
  \]
  于是, $\{ G_{\alpha, i} \mid \alpha \in A, i \in \mathbb{N} \}$是
  $\{ U_\alpha \}_{\alpha \in A}$的$\sigma$-局部有限开加细. 所以$X$是仿紧的.

  (1) $\Rightarrow$ (2) :
  因为$X$是仿紧的, 所以任意开覆盖存在局部有限开加细.
  由引理\ref{lemma:local finite close union}, 局部有限子集族是闭包保持的,
  所以任意开覆盖也存在闭包保持的开加细.

  (2) $\Rightarrow$ (3) :
  这是显然的.

  (3) $\Rightarrow$ (4) :
  设$\mathscr{U}$是$X$的一个开覆盖.
  因为$X$是正则的, 所以存在$X$的开覆盖$\mathscr{V}$, 使得
  $\{ \overline{V} \mid V \in \mathscr{V} \}$是$\mathscr{U}$的加细.
  由假设, $\mathscr{V}$存在闭包保持加细$\mathscr{W}$.
  而$\{ \overline{W} \mid W \in \mathscr{W} \}$是
  $\{ \overline{V} \mid V \in \mathscr{V} \}$的闭包保持闭加细,
  所以也是$\mathscr{U}$的闭包保持闭加细.
\end{proof}

到这里, 我们就已经有了好几个(在一定分离公理下)仿紧性的等价条件了.
以后判断一个空间是不是仿紧, 也就没有必要只盯着定义来回思考了,
还可以考虑这么多个等价的条件. 路多了, 总有一条适合你.

