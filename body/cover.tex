% -*- coding: utf-8 -*-
%%%%%%%%%%%%%%%%%%%%%%%%%%%%%%%%%%%%%%%%%%%%%%%%%%%%%%%%%%%%%%%%%%%%%%%%%
% 封面、摘要的相关内容
%%%%%%%%%%%%%%%%%%%%%%%%%%%%%%%%%%%%%%%%%%%%%%%%%%%%%%%%%%%%%%%%%%%%%%%%%

%%%------------中文个人信息------------------------
%\cdegree{学士}                          %学士/硕士/博士
\cdepart{数学系}                      %院系
\csubject{统计学}                   %专业
\ctitle{仿紧空间的性质与刻画}       %题目
%\cdirection{数学分析}                 %研究方向
\cauthor{单铭铭}                        %姓名
\csupervisor{师维学}          %导师姓名
\stitle{教授}               %职称
\cdate{二零一七年五月}                  %中文日期

%--------------英文个人信息------------------------

\etitle{Properties and Descriptions of Paracompact Spaces}%标题
\edegree{Bachelor of Science}          % 学位
\edepart{Department of Mathematics}    %院系
\esubject{Statistics}                % 专业
\edirection{General Topology}         % 研究方向
\eauthor{Mit Shan}              % 姓名
\esupervisor{Professor Weixue Shi}      % 导师姓名
\estitle{Professor}
\eaddress{Department of Mathematics, Nanjing University}     % 英文单位
\edate{May, 2017}                              % 英文日期

\ngrade{2013}                                  % 年级
\snumber{131110066}                           % 学号

%-----------------中文摘要----------------------------
\cabstract{
  本文主要是我对于仿紧拓扑空间的刻画和性质的一个学习总结,
  所有结论均来自教科书和参考资料.
  主要证明在满足$T_3$公理的前提下,几个与拓扑空间仿紧性等价的条件,
  比如其中之一是:拓扑空间$X$是仿紧的,当且仅当$X$的每个开覆盖有$\sigma-$局部有限开加细.
  在完成仿紧性质的基本刻画后,给出几个仿紧空间的性质,使仿紧性不再那么神秘.
}

\ckeywords{仿紧}



%----------------英文摘要-----------------------------
\eabstract{
  This article is my learning summary of some descriptions and properties of paracompact
  spaces. All the conclusions come from textbooks and references.
  Several equivalent conditions of paracompactness under $T_3$ axiom
  will be proved. One of them is: Topological space $X$ is paracompact,
  if and only if every open cover of $X$ has $\sigma$-local-finite open refinement.
  After describing paracompactness, we will show some properties of paracompact
  space, which will uncover the mystery of paracompactness.
}

\ekeywords{paracompact}

\makecover
