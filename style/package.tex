% -*- coding: utf-8 -*-
%%%%%%%%%%%%%%%%%%%%%%%%%%%%%%%%%%%%%%%%%%%%%%%%%%%%%%%%%%%
% 引用的宏包和相应的定义
%%%%%%%%%%%%%%%%%%%%%%%%%%%%%%%%%%%%%%%%%%%%%%%%%%%%%%%%%%%
% 图形支持宏包

\usepackage{graphicx}

\usepackage{subfig}
% 支持彩色
\usepackage{color}
% eps图像
\usepackage{epsfig}

% 首行缩进宏包
\usepackage{indentfirst}

% 版面控制宏包,定义规定的版面尺寸
%\usepackage[%paperwidth=18.4cm, paperheight= 26cm,
%            paper=a4paper,
%            vmargin={3.8 cm,3.8 cm},
%            body={14.6true cm,22true cm},
%            %twosideshift=0 pt,
%            %headheight=1.0true cm
%            ]{geometry}

% 中文支持宏包
%\usepackage{CJK}

% 脚注控制
\usepackage[perpage]{footmisc}

% AMSLaTeX宏包 用来排出更加漂亮的公式

\usepackage{amsmath}
\usepackage{amsfonts}
\usepackage{amssymb}
\usepackage{cases}
\usepackage{amstext}
\usepackage{mathtools}
\usepackage{amsthm}
\usepackage{enumitem}
%\usepackage{biblatex}
% 不同于\mathcal or \mathfrak 之类的英文花体字体
\usepackage{mathrsfs}
% 更多数学符号
%\usepackage{mathabx}

% enumerate环境, 控制行距
\setlist{nosep}
% 控制enumerate编号为(1), (2), ...
\renewcommand\labelenumi{(\theenumi)}

% 定理类环境宏包,其中 amsmath 选项用来兼容 AMS LaTeX 的宏包
%\usepackage[amsmath,thmmarks]{ntheorem}

% 因为图形可浮动到当前页的顶部,所以它可能会出现
% 在它所在文本的前面. 要防止这种情况,可使用 flafter
% 宏包
%\usepackage{flafter}

%浮动图形控制宏包
%允许上一个section的浮动图形出现在下一个section的开始部分
%该宏包提供处理浮动对象的 \FloatBarrier 命令,使所有未处
%理的浮动图形立即被处理
\usepackage[below]{placeins}

%在每一断页clearpage主要对浮动图形有用
%% \usepackage{afterpage}
%% \afterpage{\clearpage}

% 图文混排用宏包
\usepackage{floatflt}

% 图形和表格的控制
\usepackage{rotating}

% tex1cm宏包,控制字体的大小
%\usepackage{type1cm}

% 控制标题的宏包
%\usepackage[sf]{titlesec}

% 控制目录的宏包
%\usepackage{titletoc}

% 处理数学公式中的黑斜体的宏包
\usepackage{bm}

%可将浮动对象放置到文件的最后
%\usepackage{endfloat}

% fancyhdr宏包 页眉和页脚的相关定义
\usepackage{fancyhdr}
%\usepackage{fancyref}

%浮动图形和表格标题样式
\usepackage{caption}
\DeclareCaptionLabelSeparator{blank}{\hspace{1em}}

% 定制表格和图形的多行标题行距
\usepackage{setspace}

% 打印当前页面格式的宏包
\usepackage{layouts}


% 使用Times字体的宏包
%\usepackage{times}

% 表格
\usepackage{array}
\usepackage{booktabs}

%双模下划线
%\usepackage[normalem]{ulem}

%关于章节名的一些显示内容,必须放在hyperref之前
%\makeatletter
%\renewcommand\appendix{\par
%  \setcounter{chapter}{0}%
%  \setcounter{section}{0}%
%  \renewcommand\chaptername{\appendixname}
%  \renewcommand\thechapter{\Alph{chapter}}}
%\makeatother

% 生成有书签的pdf及其开关

\iffalse

\def\a{true}
\ifx\a\useyap
\usepackage[ dvipdf,
             CJKbookmarks=true,
             bookmarksnumbered=true,
             bookmarksopen=true,
             colorlinks=true,
             citecolor=blue,
             linkcolor=black,
             anchorcolor=green,
             urlcolor=blue
              ]{hyperref}
        \else
         \usepackage[dvipdfm,
             CJKbookmarks=true,
             bookmarksnumbered=true,
             bookmarksopen=true,
             colorlinks=true,
             citecolor=blue,
             linkcolor=blue,
             anchorcolor=green,
             urlcolor=blue
             ]{hyperref}
             \AtBeginDvi{\special{pdf:tounicode GBK-EUC-UCS2}} % GBK -> Unicode
\fi

\fi

% 支持引用的宏包
%\usepackage{cite}
% 支持引用缩写的宏包
\usepackage{natbib}
\usepackage{hypernat}
